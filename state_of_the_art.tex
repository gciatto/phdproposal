\section{Stato dell'arte}\label{stato-dellarte}

Oggigiorno assistiamo ad un rinnovato interesse, nell'ambito dei sistemi
distribuiti e paralleli, per quanto riguarda i modelli, i linguaggi e le
tecnologie legate allo \emph{stream processing}, cioè quel
termine-ombrello sotto al quale vengono accorpati, senza valorizzarne
opportunamente le differenze, i paradigmi che si occupano di gestire
flussi di \textbf{dati} e quelli che si occumano di flussi di
\textbf{eventi}. La parola ``evento'', in informatica, è tanto
ricorrente quanto sovraccarica di significati: essa appare in maniera
trasversale rispetto ai modelli ed alle architetture, onde per cui
troppo spesso gli si associa una semantica vaga, che facilmente collassa
su quella di ``dato''.

Nei sistemi \emph{event-based}, gli eventi rappresentano avvenimenti
significativi per il sistema, come ad esempio il cambiamento di stato di
un componente o la variazione di una proprietà del sistema o di un
partecipante al sistema stesso, cui viene associato un contesto che ne
permetta l'\emph{interpretazione}. In questi sistemi, alla
\emph{percezione} dell'evento segue la produzione di una
\textbf{notifica}, ovvero una \emph{rappresentazione} sotto forma di
dato strutturato della specifica occorrenza dell'evento stesso, e la sua
\emph{propagazione} agli altri partecipanti al sistema ad esso
interessati. Spessissimo, come anche nel seguito, con la parola
``evento'' ci si riferisce in realtà alla sua notifica. Il vantaggio
primario dei sistemi \emph{event-based} risiede nell'elevato grado di
disaccoppiamento tra i partecipanti al sistema che essi permettono.
L'espressione \emph{event-stream processing} si riferisce invece alla
capacità di gestire grandi quantità di eventi, prodotti con una
frequenza prossima al \emph{real-time}, e di elaborarli in maniera
continuativa. Infine, si parla di \emph{Complex Event Processing} (CEP)
laddove si ponga la necessità di aggregare, trasformare, correlare
semanticamente, casualmente, temporalmente o spazialmente gli eventi che
si susseguono in un sistema allo scopo di individuare l'occorrenza di
eventi più complessi, identificati da pattern di eventi più semplici.
Gli eventi possono rappresentare il meccanismo con il quale ogni sistema
raggiunge i propri requisiti funzionali e, al tempo stesso, una
importante sorgente di informazione che può essere usata dal sistema
stesso per apprendere ed adattarsi dinamicamente o dall'organizzazione
utilizzatrice del sistema per meglio comprendere le proprie attività ed
i propri processi {[}cit. process mining{]}.

Viceversa, nei sistemi che si occupano di gestire flussi di
\textbf{dati}, si impiegano architetture distribuite per elaborarne moli
imponenti, al fine di produrre un modello dei dati che permetta di
analizzare l'informazione ivi contenuta, visualizzarla o sfruttarla per
fini previsionali. In questo caso, i dati possono provenire anche da
estrazioni \emph{batch} da basi di dati, il che implicherebbe una
frequenza di immissione minore rispetto ai flussi di eventi, ed il
\emph{focus} è sulla produzione, sulla calibrazione e sul mantenimento
di modelli predittivi finalizzati all'acquisizione di informazioni
rivendibili riguardanti i gusti e le abitudi di classi di utenti {[}cit.
data mining{]}.

Flussi di dati e flussi di eventi presentano spesso molti tratti in
comune, prova ne sia l'elevato numeri di teconologie che dichiarano di
supportare entrambi indistintamente, e.g.~Apache Storm, NiFi, Flink,
Ignite, etc., ma si distinguono per la differenza intrinseca che
sussiste tra eventi e dati.

Il concetto di evento è talmente generale da far sì che i sistemi
event-based presetino intersezioni e similitudini con molte altre
macro-aree attive dal punto di vista accademico e industriale.

Nelle varianti più semplici di reti di sensori \emph{wireless} (WSN),
ogni nodo della rete è un produttore di eventi \textbf{situati}, cioè
relativi ad un preciso punto dello spazio, i.e.~la posizione del
sensore. Le funzionalità di aggregazione e correlazione dei sistemi CEP
ben si conciliano con gli scenari applicativi tipici delle WSN, come il
monitoraggio di ambienti sia artificiali (e.g. \emph{data-centers}) che
naturali (e.g.~inquinamento dell'aria, dei mari, situazione
idrogeologica, etc), in cui l'identificazione di pattern di eventi
aggregati in maniera decentralizzata risulta assi utile.

L'\emph{Internet of Things} (IoT) è un altro ambito che presenta una
naturale affinità con i sistemi ad eventi: in questa visione si immagina
che gli oggetti di uso comune (e.g.~luci, elettrodomestici, porte,
finestre, automobili, telefoni, etc) siano tutti dotati di
un'interfaccia di rete che permetta loro di interagire e cooperare per
venire incontro (o addirittura prevenire) le esigenze dei loro
utilizzatori. Non è difficile immaginare una chiave di lettura in cui
ogni ``cosa'' diventa al tempo stesso produttore e consumatore (in
inglese \emph{prosumer}, da \textbf{pro}\emph{ducer} +
\emph{con}\textbf{sumer}) di eventi, relativi alle attività compiute
dagli esseri umani. Queste attività devono essere opportunamente
\emph{interpretate} affinché il sistema possa mettere in campo una
strategia per supportare la quotidianità degli utenti, attuata per mezzo
delle ``cose'' stesse. Risulta chiara l'importanza di riuscire a dare
significato agli eventi prodotti dall'interazione tra gli utenti,
l'ambiente e le ``cose'': questi vanno interpretati sia singolarmente
che previa aggregazione temporale, spaziale e causale. Dalla pressoché
infinita varietà di situazioni possibili derivano anche la necessità di
un'interpretazione \textbf{semantica} (e non meramente sintattica) degli
eventi unita alla possibilità per il sistema di imparare a riconoscere,
dalle azioni degli utenti, le loro \textbf{attività} ed i legami di
consequenzialità, mutua esclusione, etc. tra attività diverse di una o
più persone.

La capacità di interpretare gli eventi come semanticamente correlati a
partire dalla loro rappresentazione (che si suppone contenga il nome
dell'evento in linguaggio naturale), presuppone a sua volta la facoltà
per i partecipanti al sistema --- che quindi possono essere considerati
a buon titolo agenti intelligenti {[}cit.{]} --- di avere accesso ad una
rappresentazione formale delle relazioni semantiche tra termini (e.g.
``allarme!'' è sinonimo ``pericolo!'') e di poter quindi mettere in
campo il ragionamento simbolico per interpretare gli eventi. Basi di
conoscenza che codifichino in maniera formale le relazioni tra i termini
di un particolare dominio del discorso, sono dette ontologie {[}cit.?{]}
ed esistono degli strumenti per i) esprimerle (e.g.~OWL {[}cit.{]}), ii)
produrle (e.g.~Protégé {[}cit.{]}) iii) apprenderle tramite Text-Mining.
